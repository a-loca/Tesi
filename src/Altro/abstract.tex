\thispagestyle{plain}
\chapter*{Abstract}
La composizione fotografica è un aspetto che spesso viene trascurato ai fini della valutazione della bellezza estetica delle immagini e realizzazione delle stesse da parte di fotografi amatoriali. La modalità con la quale il soggetto e gli elementi ad esso circostante vengono posizionati nello scatto può cambiare completamente il suo aspetto, la sua capacità narrativa, le sensazioni che esso riesce a trasmettere: la simmetria crea armonia e bilanciamento, elementi verticali, tendenti al cielo, trasmettono potenza e forza, profili curvi comunicano dinamismo e movimento, e così via.

Essendo un fondamentale di cui difficilmente il fotografo occasionale è a conoscenza, l'intelligenza artificiale potrebbe essere uno strumento utile ad assisterlo in questo compito e migliorare immediatamente la qualità dei suoi scatti. Attualmente, nei software delle fotocamere dei nostri smartphone esistono già strumenti che ci suggeriscono in tempo reale quale sia lo scatto migliore da effettuare. Questa funzionalità potrebbe essere ancora più potenziata introducendo anche il riconoscimento delle caratteristiche compositive di cui si discuterà in questa relazione.
 
Recenti sviluppi nel campo del \textbf{self-supervised learning} hanno portato alla nascita di DINOv2, un foundational model di MetaAI in grado di produrre features universali utili in un grande range di task a livello visivo (\textit{image classification}, \textit{instance retrieval}, \textit{video undestanding}) e a livello di pixel (\textit{semantic segmentation}, \textit{depth estimation}). 

L'obiettivo di questa tesi è condurre uno studio delle più recenti tecniche di self-supervision nell'ambito della classificazione delle immagini ed esplorare la possibilità di impiegare questo tipo di approccio nel task di classificazione della composizione fotografica, valutandone il confronto con le metodologie attualmente utilizzate nello stato dell'arte per il medesimo problema.